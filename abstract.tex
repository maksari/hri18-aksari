\begin{abstract}
\textbf{Objective:} The goal of this study is to understand how task interdependence, homogeneity, and worker characteristics, particularly sex, affect psychological outcomes of human-robot teaming for human workers. 
Background: Collaborative robots are increasingly teamed up with humans to improve productivity and task ergonomics. To ensure that humans have a positive experience collaborating with robots, we need to develop an understanding of how different task allocations based on task interdependence and homogeneity affect their experience.\\
\textbf{Method:} Thirty-two participants collaborated with a robotic arm on completing a manufacturing task. In a between-participants design, each participant was assigned to a task allocation strategy corresponding to one of four experimental conditions: (1) non-homogeneous, non-interdependent; (2) non-homogeneous, interdependent; (3) homogeneous, non-interdependent and (4) homogeneous, interdependent.  Participants\'\ perception of the robot and the collaboration was measured through subjective questionnaires as well as an semi-structured interview conducted after each experiment.\\
\textbf{Results:} When human and robot worked interdependently towards completing a task,  human had a more positive attitude towards the robot and their collaboration. Working on non-specialized, homogeneous tasks with the robot reduced human\'s attitude towards the robot and the collaboration. Task characteristics had differential effects on how men and women perceived the robot and the collaboration.
Conclusion: Task interdependence and homogeneity as well as human sex can play a major role in a human worker\'s experience collaborating with a robot on a manufacturing task.\\
\textbf{Application:} Our conclusion supports integrating task interdependence and homogeneity as variables when allocating tasks to human-robot teams in close-proximity manufacturing environments. These results could be used by designers of collaborative robots and engineers and/or software that plan and allocate tasks to human and robot.

%\footnote{This is an abstract footnote}
\end{abstract}