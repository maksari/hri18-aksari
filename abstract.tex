\begin{abstract}
The goal of this study is to understand how task and worker characteristics affect psychological outcomes of human-robot teaming for human workers. Collaborative robots are increasingly teamed up with humans to improve productivity and task ergonomics. Ensure that humans have a positive experience collaborating with robots requires an understanding of how different task allocations that result in different task characteristics affect their experience. In a laboratory study using a simulated manufacturing workcell, we manipulated the task characteristics of a manufacturing task, particularly task interdependence and homogeneity, creating four task-allocation strategies. Thirty-two participants collaborated with a robot arm to complete a manufacturing task following one of these strategies, and their perceptions of the robot and of the collaboration were measured through questionnaires. Our results showed that when human and robot workers performed the task interdependently, participants had a more positive attitude toward the robot and their collaboration. Working on tasks homogeneously with the robot reduced positive attitudes toward the robot and the collaboration. Task characteristics had differential effects on how men and women perceived the robot and the collaboration. Our findings highlight the need to consider task characteristics when allocating tasks to human-robot teams in close-proximity manufacturing environments and inform designers of collaborative robots and engineers engaged in task planning for human-robot teams.

\end{abstract}