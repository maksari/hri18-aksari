\subsection{Limitations}
In this paper, we investigated the effects of task homogeneity and interdependence on human perceptions of a robotic collaborator in a manufacturing task. Other task characteristics, such as the relative social statuses of the human and robot workers or inherent differences in performance or competence between the human and robot workers, might shape worker experience and can be studied in future research. In our experiments, participants collaborated with a robot in a relatively simple manufacturing task that did not involve complex dependencies in tool or workspace use between the workers. Understanding how the studied task characteristics affect human-robot collaboration under different levels of task complexity requires further investigation. Additionally, participants in our study were engaged in a brief manufacturing task that was simulated in a laboratory environment, and establishing the generalizability of our findings to real-world manufacturing settings and long-term human-robot collaborations requires further research. Furthermore, the manipulations in task characteristics resulted in different task allocations for the human and the robot worker, which prevented us from making meaningful comparisons of task performance and behavior across conditions. Finally, we expect the design of the specific robot platform used to have an effect on worker perceptions. The Kinova Mico arm was designed as a lightweight multipurpose arm to be integrated into day-to-day human environments, and thus collaborative robots designed for industrial environments might elicit different perceptions.
