\section{Discussion}
We hypothesized that when human and robot work were interdependent, human workers would show a more positive attitude toward robot and the resulting collaboration than when their work were independent. Consistent with our prediction and insights offered by recent literature (e.g., \cite{johnson2012autonomy}), our results showed that task interdependence improved participant experience with the robot, particularly their perceptions of the robot's competence and the robot as a collaborator. When participant work depended on the robot's work, participants perceived the robot to be more competent and to be a more effective collaborator. We speculate that this effect is partly due to the interdependent task requiring participants to closely observe and analyze the robot's work and gain insight into the robot's operation. Furthermore, task interdependency might have resulted in a stronger affiliation between the participant and the robot, resulting in perceptions of the robot to be on a more equal footing and thus more positive. This explanation is also supported by prior work in human-computer and human-robot interaction that has suggested people perceive computers and robots that are similar to themselves more positively than those that are different from themselves \cite{nass2000does,andrist2015look}.

We also hypothesized that collaborating on specialized, non-homogeneous tasks with robots would improve attitudes toward robotic teammates and the resulting collaboration compared to collaborating on non-specialized, homogeneous tasks. Our results indicated that task homogeneity affects participants' perceptions of the robot. Specifically, when participants worked on different tasks than those of the robot, they perceived the robot as being more competent. This finding is consistent with results reported by research on job design on the effects of task specialization on worker satisfaction \cite{hsieh2004reassessment}, suggesting that human workers may perceive robotic collaborators to be more competent when task allocation requires each worker to specialize and offer unique contributions aligned with their specialization.

Our exploratory analysis highlighted effects of worker sex on worker perceptions of the task and complex ways in which worker sex interacted with task characteristics. In particular, task homogeneity decreased perceived task load among women but increased it among men. One potential explanation of this interaction is that task homogeneity was  perceived to create competition between the participant and the robot. As discussed in the Background Section, prior work in human-robot interaction suggest differential perceptions of robotic partners among men and women under competitive and cooperative task structures, such that women report a more positive experience when they cooperate with a robot while men report a more positive experience when they compete with the robot \cite{mutlu2006task}. In the current study, men might have perceived homogenous tasks to be more competitive and thus to involve higher task load, and task homogeneity might have promoted a more collaborative environment and thus reduced task load for women.

We also found that previous familiarity with gaming, robots, and manufacturing positively predicted participant experience and negatively predicted task load. This finding suggests that individuals with greater experience with interacting with advanced technology and with manufacturing might be more open and adept to collaborating with robots. 