\section{Discussion}
        We hypothesized that when human and robot’s work is interdependent, human workers will have a more positive attitude toward robots and the resulting collaboration than when their work is independent. Consistent with our prediction and that offered in recent literature (e.g., Johnson et al., 2012), our results showed that task interdependence improved participant experience with the robot, particularly their perceptions of the robot’s competence and the robot as a collaborator. When participant work depended on the robot’s work, participants perceived the robot to be more competent and to be a more effective collaborator. We speculate that this effect is partly due to the interdependent task requiring participants to closely observe and analyze the robot’s work and gain insight into the robot’s operation. Furthermore, task interdependency might have resulted in a stronger affiliation between the participant and the robot, resulting in perceptions of the robot to be on a more equal footing and thus more positive. 
       We also hypothesized that collaborating on non-specialized homogeneous tasks with robots will reduce attitudes towards robotic teammates and the resulting collaboration compared to collaborating on specialized non-homogenous tasks. Our results indicated that task homogeneity affects participants’ perceptions of the robot. Specifically, when participants worked on different tasks than those of the robot, they perceived the robot as more competent. This finding is consistent with results reported by research on job design on the effects of task specialization on worker satisfaction, suggesting that human workers may perceive robotic collaborators to be more competent when task allocation requires each worker to provide unique specialization.
       Our exploratory analysis highlighted effects of worker sex on worker perceptions of the task and complex ways in which worker sex interacted with task characteristics. In particular, homogeneity decreased task load among women but increased it among men. One potential explanation of this interaction is that task homogeneity was be perceived to create competition between the participant and the robot. Prior work in human-robot interaction suggest differential perceptions of robotic partners among men and women under competitive and cooperative task structures, such that women report a more positive experience when they cooperate with a robot while men report a more positive experience when they compete with the robot (Mutlu et al., 2006). In the current study, men might have perceived homogenous tasks to be more competitive and thus to involve higher task load, and task homogeneity might have promoted a more collaborative environment and thus reduced task load for women. 
       Previous familiarity with gaming, robots, and manufacturing positively predicted  participant experience and negatively predicted  task load. This finding suggests that individuals with greater experience with interacting with advanced technology and with manufacturing might be more open and adept to collaborating with robots. 