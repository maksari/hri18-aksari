\section{Hypotheses}
Based on previous research on the effects of task characteristics on collaboration, we formulated the following set of hypotheses on how task interdependence and homogeneity may affect worker attitudes toward robotic teammates and their work.

\textbf{Hypothesis 1.} When humans and robots work \textit{inter}dependently, human workers' attitudes toward their robotic teammates and the resulting collaboration will be more positive than when their work is independent.

The basis of this hypothesis is findings from research on task interdependence that suggests that interdependent tasks may improve job outcomes and increase sense of responsibility for other's work \cite{van1998motivating, johnson2012autonomy}.

\textbf{Hypothesis 2.} Worker attitudes towards robotic teammates and the resulting collaboration will be more positive when collaboration involves specialized, non-homogeneous tasks compared to collaborating on non-specialized, homogenous tasks.

While task homogeneity has not been extensively explored, prior research suggests that \textit{task specialization} improves professional efficacy and thus promotes worker confidence \cite{hsieh2004reassessment}, providing the basis for this hypothesis. 

Additionally, while we did not posit specific hypotheses due to lack of prior evidence in the context of teaming in manufacturing settings, we expected individual differences, specifically worker sex, to affect attitudes toward robotic teammates and perceptions of collaborative work.
