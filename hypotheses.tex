\section{Hypotheses}
       Based on previous research on the effects of task characteristics on collaboration, we formulated following set of hypotheses on how task interdependence and homogeneity affect worker attitudes toward robotic teammates and toward their collaborative work.\\
       \textbf{Hypothesis 1.} When a human and a robot work interdependently, the human worker will have more positive attitude toward robots and the resulting collaboration than when their work is independent.\\
       The basis of this hypothesis is findings from research on task interdependence that suggests 
that interdependent tasks may improve job outcomes and increase sense of responsibility for other\' s work \cite{van1998motivating, johnson2012autonomy}.\\
       \textbf{Hypothesis 2.} Collaborating on non-specialized homogeneous tasks with robots will reduce attitudes towards robotic teammates and the resulting collaboration compared to collaborating on specialized non-homogenous tasks.\\
       While task homogeneity has not been extensively explored, prior research suggests that task specialization improves professional efficacy and thus promotes worker confidence \cite{hsieh2004reassessment}, providing the basis for this hypothesis.\\
       Additionally, while we did not posit specific hypotheses due to a lack of prior evidence, we expected individual differences, specifically worker sex, to affect attitudes toward robotic teammates and perceptions of collaborative work.
