\section{Results}
       To ensure the reliability of the user experience questionnaire, we conducted a confirmatory factor analysis and eliminated items that showed poor consistency with the other items in each scale. This analysis resulted in a four-item scale of competence (items 1, 4, 5, 7; Cronbach\' s $\alpha$ = 0.76), a two-item scale of contribution (items 19, 21; Cronbach\' s $\alpha$ = 0.9), a two-item scale of enjoyment (items 23, 24; Cronbach\' s $\alpha$ = 0.75),  a two-item scale of collaboration (items 23, 28; Cronbach\' s $\alpha$ = 0.72), and a five-item scale of task load (items 8,9,10,12, 15; Cronbach\' s $\alpha$ = 0.85). The analysis showed that the items of the presence scale did not form a reliable scale, and thus this scale was eliminated from analysis.\\
       Following the reliability analysis, we conducted an analysis of covariance (ANCOVA), including task homogeneity and interdependence as fixed effects and participant sex as a covariate for each measure. Our first hypothesis was that when human and robot’s work is interdependent, human workers will have a more positive attitude toward robots and the resulting collaboration than when their work is independent. Our results confirmed this hypothesis. Our  analysis showed that when human and robot work depended on each other, participants rated the robot as being more competent, F(1, 28) = 6.54, p = 0.0187, and perceived it more as a collaborator, F(1, 28) = 3.73, p = 0.067, compared to when there was no interdependency. 

Our second hypothesis was that collaborating on non-specialized homogeneous tasks with robots will reduce attitudes towards robotic teammates and the resulting collaboration compared to collaborating on specialized non-homogeneous tasks. Confirming our hypothesis, we found that when participants and the robot worked on homogeneous tasks, they rated the robot\' s competence lower compared to when they worked on non-homogeneous tasks, F(1, 28) = 9.74, p = 0.0042. 
       As noted above, we also conducted an exploratory analysis of the effects of worker sex on worker experience, as prior work in human-robot collaboration point toward significant differences in the attitudes of men and women toward robots \cite{schermerhorn2008robot}, which we suspect to observe in the manufacturing setting. Our analysis revealed that task characteristics had a differential effect on how men and women perceived of the robot and the task. For example, female participants found the robot to be more cooperative than male participants did and were more willing to collaborate with the robot than males were, under the condition when their task depended on the robot\' s, F(1,28) = 4.18, p = 0.054. Similarly, female participants found the robot to be more competent when their work depended on that of the robot, F(1,28) = 5.16, p = 0.032. We also found an interaction effect between participant sex and task homogeneity over perceptions of the task: male participants found the task to be more mentally and physically demanding when task allocation was homogeneous whereas female participants found it to be more demanding when it was not homogeneous, F(1,28) = 6.98, p = 0.0156.\\
       Finally, we conducted additional exploratory analyses, replacing participant sex with other demographic factors, in order to identify other individual differences future research might investigate. Results from this analysis showed that the frequency with which participants played computer games predicted perceived competence of robot, F(1,28) = 7.83, p = 0.011), willingness to collaborate with the robot, F(1,28) = 10.39, p = 0.004, and perceived contribution of the robot, F(1,28) = 7.25, p = 0.014 negatively predicted task load, F(1,28) = 7.69, p = 0.0117. Participant familiarity with manufacturing predicted perceived competence of the robot, F(1,28) = 7.54, p = 0.0124, and familiarity with robots predicted willingness to collaborate with the robot, F(1,28) = 5.27, p = 0.032.