\section{Results}
Our first hypothesis predicted that when human and robot work were interdependent, human workers would have a more positive attitude toward robots and the resulting collaboration than when their work were independent. Our analyses provided support for this hypothesis; we found that when human and robot work depended on each other, participants rated the robot as being significantly more competent, $F(1, 28) = 6.54$, $p = 0.0187$, and perceived it to be marginally more of a collaborator, $F(1, 28) = 3.73$, $p = 0.067$, compared to when there was no interdependency. We found no differences across different levels of task interdependence in other measures.

Our second hypothesis was that participant attitudes toward robotic teammates and the resulting collaboration would be more positive when collaborating on specialized, non-homogeneous tasks than in collaborations on non-specialized, homogeneous tasks. Consistent with our hypothesis, we found that when participants and the robot worked on non-homogeneous tasks, they rated the robot to be significantly more competent compared to when they worked on homogeneous tasks, $F(1, 28) = 9.74$, $p = 0.0042$. Data from other measures provided did not vary across different levels of task homogeneity.

As noted in the Hypothesis Section, we also conducted an exploratory analysis of the effects of worker sex on worker experience, as prior work in human-robot collaboration point toward significant differences in the attitudes of men and women toward robots \cite{mutlu2006task,mutlu2006storytelling,schermerhorn2008robot,takayama2009influences}, which we expected to observe in the manufacturing setting. This exploratory analysis revealed that task characteristics had a differential effect on how men and women perceived of the robot and the task; female participants were marginally more willing to collaborate with the robot than males were when their work depended on the robot's work, $F(1,28) = 4.18$, $p = 0.054$. Similarly, female participants found the robot to be more competent when their work depended on that of the robot, $F(1,28) = 5.16$, $p = 0.032$. We also found an interaction effect between participant sex and task homogeneity over perceptions of the task load, $F(1,28) = 6.98$, $p = 0.0156$; male participants found the task to be more mentally and physically demanding when task allocation was homogeneous whereas female participants found it to be more demanding when it was non-homogeneous.

Finally, we conducted additional exploratory analyses, replacing participant sex with other demographic factors, in order to identify other individual differences future research might investigate. Results from this analysis showed that the frequency with which participants played computer games significantly predicted perceived competence of robot, $F(1,28) = 7.83$, $p = 0.011$, willingness to collaborate with the robot, $F(1,28) = 10.39$, $p = 0.004$, and perceived contribution of the robot, $F(1,28) = 7.25$, $p = 0.014$, while negatively significantly predicting task load, $F(1,28) = 7.69$, $p = 0.0117$. Participant familiarity with manufacturing significantly predicted perceived competence of the robot, $F(1,28) = 7.54$, $p = 0.0124$, and familiarity with robots significantly predicted willingness to collaborate with the robot, $F(1,28) = 5.27$, $p = 0.032$.