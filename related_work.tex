\section{Related Work}
       As human-robot teams are becoming more prevalent, in some cases work that was previously done by a human-human team is assigned to these new human-robot teams. Therefore, we draw on a strong body of literature that has studied human-human teams in various settings as well as more recent works that study human-robot teams. In particular, below we review literature on how structural properties of tasks can affect team outcome, how humans perceive their collaborators in a team, and how individual human differences can affect attitudes towards a collaborative robot.
\subsection{Task Characteristics and Collaboration}
       Structural properties of tasks assigned to a person can affect perception of that person of the task and or his collaboration with the team. Hackman and Oldham \cite{hackman1976motivation} developed a Job Characteristics Model consisting of five principles: task identity, task significance, skill variety, job feedback, and autonomy. These principles can affect the psychological state of the workers in how much they learn, how much they feel responsible for the outcomes of the works, and how much they care about the work outcome. \\
       Task interdependence is one such key property of teamwork. Kiggundu \cite{kiggundu1983task} found that employees respond positively to task interdependence when the setup of the task involve providing resources to others or necessary for others\' success. This work found no negative relationship between interdependence and job outcomes. In fact, interdependence appears to improve a sense of responsibility and personal  work outcomes \cite{van1998motivating}. Furthermore, the level of task interdependence in a team can affect team outcomes \cite{katz2005collective, langfred2005autonomy, liden1997task}. As the level of interdependence changes, different methods of communication and coordination between members are needed for optimal team performance \cite{espinosa2004explicit, butchibabu2016implicit}.\\
       Job specialization is another property that affects individual and consequently team performance. Traditionally, manufacturing industries have employed job rotation as a means to mitigate feelings of monotony, boredom, and mental fatigue in workers \cite{miller1973job,kuijer1999job}. However, the emergence of advanced technology and robotics has changed the relationship between job rotation and job burnout.  Recent studies in the automotive industry suggest that workers no longer prefer job rotation, but they instead prefer to work on their own specialized tasks to increase their level of professional efficacy and lower feelings of replaceability \cite{hsieh2004reassessment}. \\
       The type and structure of the task, such as whether or not the task involves manual work and the roles that individual workers play, can also affect job outcomes. Stewart et al. \cite{stewart2000team} find that task interdependence can have a very different effect on team outcome when working on conceptual versus behavioral tasks. These factors also appear to affect human-robot teams. For example, when teaming with a subordinate robot, human collaborators feel more responsibility toward the outcome \cite{hinds2004whose}. Furthermore, people prefer robots whose appearance matches the type of the task assigned to the robot \cite{goetz2003matching}.\\
       The effects of task characteristics such as interdependence also determine the success of human-robot teams. Johnson et al. \cite{johnson2012autonomy} argued that robots must be designed such that they can work interdependently with humans and that, otherwise, team performance will suffer under complex situations where dependencies exist. Hinds et al. \cite{hinds2004whose} provide evidence that task interdependence affects a sense of responsibility human workers feel toward collaborative work. Nikolaidis et al. \cite{nikolaidis2012human, nikolaidis2013human} have explored how interdependent work can be facilitated  by enabling the development of shared mental models through human-robot cross-training. 
\subsection{Perceptions of Collaborator}
       When people work in teams, they develop perceptions of the work and traits of their collaborators, such their ability and integrity, and these perceptions can change over the course of their collaboration \cite{jarvenpaa1994global}. These perceptions extend to human-computer and human-robot teams. For example, when computers interdependently work on a task with humans, people view them as being more similar to themselves, more friendly, and more cooperative \cite{nass1996can}. When collaborating with robots, the behavior of the robot, such as whether the robot makes anticipatory decisions, can affect perceived contribution of the robot to the team\' s success \cite{hoffman2007effects} and its awareness of its human counterpart \cite{huang2015adaptive}.
\subsection{Individual Differences}
       Research in human-robot interaction and teaming also show that individual differences can affect people\' s attitudes towards robots. For example, people with natural science and technology backgrounds were found to show a more positive attitude towards the robot compared to those with backgrounds in the social sciences \cite{nomura2011exploring}. Another study found that, in an arithmetic task in the presence of a robot, males saw the robot more human-like and more socially-desirable than females did \cite{schermerhorn2008robot}.