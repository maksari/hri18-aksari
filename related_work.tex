\section{Related Work}
As collaborative-robot technology becomes more prevalent, work that was previously done by a \textit{human-human} team is being reformulated for \textit{human-robot} teams. Therefore, we draw on a strong body of literature that has studied human-human teams in various settings as well as more recent research on human-robot teaming. In particular, below we review literature on how structural properties of tasks can affect team outcomes, how humans perceive their collaborators in a team, and how individual differences can affect attitudes toward working with a collaborative robot.
       
\subsection{Task Characteristics \& Collaboration}
Structural properties of tasks assigned to a person can affect the perceptions of that person of the task and or his collaboration with his/her team. \citet{hackman1976motivation} developed a \textit{Job Characteristics Model} consisting of five properties---task identity, task significance, skill variety, job feedback, and autonomy---and suggested that these properties can affect the workers in how much they learn, how responsible they feel for the outcomes of the work, and how much they care about the work outcome.

\textit{Task interdependence} is one such key property of teamwork. \citet{kiggundu1983task} suggests that employees respond positively to task interdependence when the setup of the task involves providing resources to others that are necessary for others' success. This work found task interdependence to improve a sense of responsibility and personal work outcomes \cite{van1998motivating}. The \textit{level} of task interdependence in a team has also been found to affect team outcomes \cite{katz2005collective, langfred2005autonomy, liden1997task}. As the level of interdependence changes, different methods of communication and coordination between members are needed for optimal team performance \cite{espinosa2004explicit, butchibabu2016implicit}.

\textit{Job specialization} is another property that affects individual and consequently team performance. Traditionally, manufacturing facilities have employed \textit{job rotation} as a means to mitigate feelings of monotony, boredom, and mental fatigue in workers \cite{miller1973job,kuijer1999job}. However, the emergence of advanced technology, including manufacturing robotics, has changed the relationship between job rotation and job burnout. Recent studies in the automotive industry suggest that workers no longer prefer job rotation, but they instead prefer to work on their own specialized tasks to increase their level of professional efficacy and lower feelings of replaceability \cite{hsieh2004reassessment}.

The type and structure of the task, such as whether or not the task involves manual work and the roles that individual workers play, can also affect job outcomes. \citet{stewart2000team} found that task interdependence can have a differential effect on team outcome when working on conceptual versus behavioral tasks. These factors also appear to affect human-robot teams. For example, when teaming with a subordinate robot, human collaborators feel more responsibility toward the outcome \cite{hinds2004whose}. Furthermore, people prefer robots whose appearance matches the type of the task assigned to the robot \cite{goetz2003matching}.

The effects of task characteristics such as interdependence also determine the success of human-robot teams. \citet{johnson2012autonomy} argued that robots must be designed such that they can work interdependently with humans and that, otherwise, team performance would suffer under complex situations where dependencies exist. \citet{hinds2004whose} provide evidence that task interdependence affects the sense of responsibility that human workers feel toward collaborative work. Nikolaidis et al. \cite{nikolaidis2012human, nikolaidis2013human} have explored how interdependent work can be facilitated by enabling the development of shared mental models through human-robot cross-training.

\subsection{Collaborator Perceptions}
When people work in teams, they develop perceptions of the work and traits of their collaborators, such as their ability and integrity, and these perceptions can change over the course of their collaboration \cite{jarvenpaa1994global}. These perceptions extend to human-computer and human-robot teams. For example, when computers interdependently work on a task with humans, people view them as being more similar to themselves, more friendly, and more cooperative \cite{nass1996can}. When collaborating with robots, the behavior of the robot, such as whether the robot makes anticipatory decisions, can affect perceived contribution of the robot to the team's success \cite{hoffman2007effects} and its awareness of its human counterpart \cite{huang2015adaptive,huang2016anticipatory}.

\subsection{Individual Differences}
Research in human-robot interaction and teaming has also shown that individual differences can affect people's attitudes towards robots. For example, people with backgrounds in science and technology were found to show a more positive attitude toward the robot compared to those with backgrounds in the social sciences \cite{nomura2011exploring}. Several other studies found differences in how men and women perceive robotic assistants, partners, and collaborators \cite{mutlu2006storytelling,schermerhorn2008robot,takayama2009influences}. Among these studies, \citet{mutlu2006task} investigated how men and women perceived a robotic partner under different task structures, comparing cooperative and competitive structures. They found significant differences in how socially desirable men found their robotic partner to be and their overall positive affect across different task structures. They found no effects of task structure on women's experiences and perceptions of the robot. These studies suggest fundamental differences in how men and women perceive teammates when task characteristics change and motivate further investigation in the context of task-based human interaction with collaborative robots.