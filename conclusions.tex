\section{Conclusion}
Collaborative robots are increasingly teamed up with humans in a wide-range of settings including shop floors and small-batch-manufacturing facilities. To ensure their successful integration into these environments and their acceptance by human workers, we must develop a better understanding of how to best allocate tasks between human and robot co-workers and how the characteristics of resulting tasks affect worker experience, satisfaction, and perceptions of robotic collaborators. In this paper, we studied two variables for task allocation, \textit{task interdependence} and \textit{task homogeneity}, and investigate how these two characteristics affect human experience with robots and with collaborating with robotic workers. We also analyzed how worker sex interacted with these characteristics. We conducted a $2\times2$ between-participants study that manipulated task interdependence and task homogeneity, which resulted in four unique experimental conditions. Based on previous work that studied task interdependence and job specialization in  human-human teams as well as human-robot teams, we predicted human workers to have a more positive attitude toward their robotic collaborators and toward their work (1) when human and robot work were interdependent than when their work were independent and (2) when they worked in specialized, non-homogeneous tasks than when they collaborated over non-specialized, homogeneous tasks. Consistent with these hypothesis, we found that participants perceived the robot to be more competent and were more willing to collaborate with the robot when their tasks were interdependent. We also found that they rated the robot to be more competent when working on a specialized, non-homogeneous task with the robot. Furthermore, worker sex interacted with task homogeneity and interdependence in complex ways. The findings of this study illustrate the significance of considering task and worker characteristics for task allocation in human-robot teams.