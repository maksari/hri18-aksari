\section{Conclusions}
       Collaborative robots are increasingly teamed up with humans to complete a variety of tasks. Given the diverse capabilities of these robots, we need to study how to best allocate tasks between human and the robot co-worker. In this paper, we study two variables for task allocation, task interdependence and task homogeneity, and investigate how these two properties affect human experience with the robot and with the collaboration. We also analyze how worker sex interacts with these properties. We conducted a $2\times2$ between-participants study where we manipulated task interdependence and task homogeneity, which resulted in four unique experimental conditions. Based on previous work that studied task interdependence and job specialization in  human-human teams as well as human-robot teams, we formulated two hypotheses: 1) When human and robot work is interdependent, human workers will have a more positive attitude toward robots and the resulting collaboration than when their work is independent. and 2) Collaborating on non-specialized homogeneous tasks with robots will reduce attitudes towards robotic teammates and the resulting collaboration compared to collaborating on specialized non-homogeneous tasks. Consistent with our hypothesis, we found that participants perceived the robot more competent and were willing to collaborate more with the robot when task interdependence existed. We also found that they rated the robot less competent when working on a homogeneous task  with the robot. Furthermore,  worker sex interacted with task homogeneity and interdependence in complex ways. The findings of this study illustrate the significance of considering task interdependence and homogeneity  as well as worker sex for task allocation in human-robot teams.