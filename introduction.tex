\section{Introduction}
       Advanced robotic technology has opened up the possibility of integrating highly autonomous robots into shared workspaces with human teams.  These collaborative robots are envisioned to increase productivity of human labor, allow greater flexibility in production, and improve ergonomics of manual tasks \cite{peshkin1999cobots, tan2009human}.  However, this capability raises the question of how to best allocate work to maximize team efficiency and the experience of human team members working with their robotic teammates.\\
       While prior work on "task allocation" in human-robot teams, or "human-robot teaming", has explored methods for allocating work for minimizing task completion time (makespan) and ergonomic impact \cite{shah2011improved, gombolay2014decision, tsarouchi2017ijcim}, other factors exist that can affect worker experience with collaborating with robots. Positive worker experience can in turn improve worker job satisfaction and lead to successful adoption of new technology. One such factor is task homogeneity - whether we assign tasks  such that a robot and its human collaborator work on different, specialized set of tasks (i.e. non-homogenous) or such that they work on the same set of non-specialized tasks (i.e. homogeneous). When task homogeneity is high, human and robot workers share workspace, tools, and supplies, which may increase the need for coordination. Another factor that may affect worker experience in task allocation is task interdependence \cite{kiggundu1983task} - whether we allocate tasks such that human and robot workers depend on each other for completing a task or work in parallel without any dependency. Higher task interdependence requires increased coordination between human-robot teammates.  In this study, we test the main effects and the interaction effects of task homogeneity and interdependence on the human workers\' perception of the robot as well as the effectiveness of the team collaboration. While prior work on human-only teams has explored how job specialization and interdependence can affect worker\' s self-efficacy, job satisfaction, and team outcomes [Miller, 1973; Kuijer, 1999; Hsieh 2004], it remains unclear how these factors translate when collaborating with a robot.  Furthermore, worker characteristics may add to this complexity. For example, human-robot interaction literature highlight stark differences in how men and women interact with robots  \cite{schermerhorn2008robot}, suggesting that sex may be one of the factors that shape worker experience with collaborative robots.\\
       In this paper, we study how task interdependence, homogeneity, and worker characteristics, particularly sex, affect psychological outcomes of human-robot teaming for human workers, such as perceptions of robotic teammates, perceptions of collaboration, and perceptions of work. The results of this study can guide us in more effective task allocation for human-robot teams.
